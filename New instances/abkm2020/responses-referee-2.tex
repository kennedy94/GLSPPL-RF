\documentclass[11pt]{article}

\usepackage{amsmath}
\usepackage{amssymb}
\usepackage{graphicx}
\usepackage{xcolor}
\usepackage{url}

\setlength{\textwidth}     {16.0cm}
\setlength{\textheight}    {21.0cm}
\setlength{\evensidemargin}{ 0.5cm}
\setlength{\oddsidemargin} { 0.5cm}
\setlength{\topmargin}     {-0.5cm}
\setlength{\baselineskip}  { 0.7cm}

\newcommand{\Om}{\Omega}
\newcommand{\bx}{\boldsymbol{x}}
\newcommand{\tred}{\textcolor{red}}

\parindent0cm

\begin{document}

\begin{center}
{\Large Responses to Reviewer~\#2}
\end{center}

\vspace{1cm}

The authors would like to thank the careful reading of the reviewer, whose comments helped  to improve the quality of the manuscript. All points raised by the referee were taken into account as described below. The corresponding modifications in the revised version appear in \textcolor{blue}{blue}.

\noindent
\textbf{\Large }

\begin{description}
  
\item[\textbf{General comment:}] \textit{The authors presented a mathematical model for an integrated lot sizing and scheduling problem in a large company in the personal care consumer goods industry. They developed several problem-dependent and problem-independent strategies based on the relax-and-fix heuristic for their model. The performance of the heuristics was evaluated by solving randomly generated instances and real-world cases. The manuscript is an interesting problem and fits the aims of the journal and also it is pertinent for the journal's readership. However, I have the following major concerns.}

\item[\textbf{Answer:}] Thank you very much for your positive review.

\item[\textbf{Remark 1:}] \textit{The paper requires careful language editing (e.g., usage of articles, clauses, punctuation). The authors should work on the logic flow of Section 1 to improve clarity.}

\item[\textbf{Answer:}] The entire manuscript has been revised for language editing. Section~1 has been reworded to improve its logical flow.

\item[\textbf{Remark 2:}] \textit{Managerial implications and insights are missing from the paper. Please emphasize the applicability of your model and heuristics. How easy is it to implement them in practice? Are there any obstacles and limitations that could restrict implementing your model and strategies?}

\item[\textbf{Answer:}] From a technical standpoint, companies use three types of solutions: (i) market-ready tools like SAP Advanced Planning and Optimization (SAP-APO), (ii) off-the-shelf optimization software like CPLEX to solve customized problems, and (iii) development of ad-hoc solution methodologies like heuristic methods and metaheuristics to solve specific problems.

From a managerial standpoint, when a tool is available on the market, such as in options (i) and (ii), the organization feels more comfortable with making decisions. \textcolor{red}{Additionally} Still, the following advantages stand out in scenario (i): user-friendly interface, database integration, standardization in the settlement of the company's various problems, ease of interpretation of results, and technical support provided by the software developer. Option (i) has the disadvantage that the company must adapt to the software's capabilities and is unable to address particular challenges. In scenario (iii), on the other hand, the corporation can handle, using ad-hoc methods, problems that more accurately reflect the organization's actual problems. The downside of option (iii) is that the development of specific techniques takes more time and requires the hiring of specialized teams. There is also the risk of not being able to continue using the tool if there is a discontinuity in the technical team that developed the project. Option (ii) is a compromise between the other two solutions, as it is possible to handle company-specific problems with little technical knowledge and low expense. The proposal in this work fits into this category. This proposal demonstrates how a genuine production planning problem in a cosmetic and personal care company was handled using basic heuristics that make use of commercial software for tackling integer programming problems.

To summarize, the proposed strategy has the advantages of being simple to implement with minimum technical knowledge and being based on well-established software that solves subproblems. This offers the company confidence in its decision-making. On the downside, it does not integrate with databases, and the findings are not as straightforward to analyze as those provided by market-ready solutions. The paper compares the solutions generated by the suggested approach with the solutions now utilized by the company, which uses a market-ready product, to demonstrate the financial benefits that using the proposed strategy can bring to the organization. As a possible conclusion, approaches like the one presented in the present work should be included in market-ready products.

\textcolor{red}{This discussion was included in the manuscript conclusions (Section 6).}

\item[\textbf{Remark 3:}] \textit{What are the contributions of your paper? I recommend comparing more features of your research with recent studies in a table to distinguish your contributions.}

\item[\textbf{Answer:}] The financial component of production planning optimization is the first thing to examine: it  enhances the company's financial results in terms of inventory reduction, storage, and manufacturing expenses. Indeed, the financial benefit usually outweighs the costs of developing and implementing the optimization method by several orders of magnitude. 

Analyzing a real case of a company in the cosmetics industry, this paper illustrates how operational research techniques can be used to optimize industrial processes. Additionally, the model considered includes as a constraint the \textcolor{red}{total} storage capacity of the warehouses, a feature that is not found in the literature. Following a suggestion of the referee, this fact is highlighted in the new Section 5.3, in which the storage capacity of the warehouses is used to perform a sensitivity analysis. 

Lot-sizing and scheduling are classic production engineering problems and the literature dealing with them separately or integrated is enormous. \textcolor{red}{For example, an extensive review can be found in Copil et al. (2017), where an overview of the literature on simultaneous lotsizing and scheduling problems is presented in a 64-page paper.}

A complete literature review would take too much space and is beyond the scope of this paper. In the paper, we attempted to highlight the contributions along the lines of what was mentioned above.


\item[\textbf{Remark 4:}] \textit{The proposed solution method was selected due to the fact that this heuristic method had been successfully applied to similar problems. It is not a good justification. What is the advantage of this heuristic method over other methods?}

\item[\textbf{Answer:}] The justification for the choice of the chosen technique was reformulated, along the lines of what was stated in the answer to remark 2.

\textcolor{red}{(Final da pg. 3 do artigo) It should be highlighted that the proposed strategy has the advantages of being simple to implement with minimal technical knowledge and being based on well-established software that solves sub-problems. The proposed strategy is a compromise between market-ready tools, where the firm must adapt to the capabilities of the software, and the development of more elaborate ad-hoc solution methodologies such as metaheuristics to solve specific problems that requires hiring specialized teams.}

\item[\textbf{Remark 5:}] \textit{It seems that 25 randomly generated instances (mentioned in Subsection 5.1) are not enough to get valid results.}

\item[\textbf{Answer:}] All experiments, tables and figures were redone, considering a set with 50 instances.From a qualitative point of view, previous conclusions were maintained.

\item[\textbf{Remark 6:}] \textit{I suggest the authors add a new section for sensitivity analyses to present more managerial insights and decision aids.}

\item[\textbf{Answer:}] One of the features considered in the present model that is not considered in similar models is the capacity of the warehouse. We use this fact as a starting point and perform a sensitivity analysis to see how the solution is affected when we vary the installed capacity of the warehouse. In fact, this type of analysis makes strategic sense because increasing or decreasing storage capacity is a simple matter -- leasing or unleasing a warehouse can typically take a month or so, while buying and installing a new machine can take one or more years. This analysis led to the new Section~5.3.

\end{description}

\end{document}
